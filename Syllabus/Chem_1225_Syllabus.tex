\documentclass[12pt, letterpaper]{article}
\usepackage{SyllabusStyle}

\begin{document}
\begin{center}
{\Large \textsc{Principles of Chemistry Lab II}}

CHEM 1225
\end{center}

\begin{center}
{\large Spring 2022}
\end{center}
\begin{center}
	\rule{0.99\textwidth}{0.4pt}
	\begin{tabular}{llcll}
		\textbf{Instructor:} & Matthew Rowley & & \textbf{Office Hours:} & Daily 10:00 am -- 11:00 am \\
		\textbf{Telephone:} & (435) 586-7875 & & & \\
		\textbf{Email:} & matthewrowley$1$@suu.edu  & & \textbf{Office:} & SC-220\\
		\multicolumn{5}{c}{Please include the course number in the subject line of all correspondence.}
	\end{tabular}
	\rule{0.99\textwidth}{0.4pt}
\end{center}

\section*{Course Description}
This is the lab to accompany CHEM 1220. A mimimum grade of ``C'' (2.0 or above) must be earned in this course before it can be counted toward a physical science major or minor or as a prerequisite for any other course.

\paragraph{Prerequisites:}
None

\paragraph{Concurrent requisite:}
CHEM 1220 -- Principles of Chemistry II

\paragraph{Course Materials:} ~

\emph{CHEM 1225 Experiments for Chemical Principles II} by Bronsema (Available at the SUU Bookstore)

You are required to bring and wear your own pair of OSHA-approved safety goggles to \emph{every} lab. Students without eye protection will be required to leave the lab and will receive a zero for the labwork that day.

\paragraph{Student Learning Outcomes:}
\begin{description}
	\item[Knowledge of the physical and natural world] -- Students will recall, interpret, compare, explain, and apply chemistry terminology and theory.
	\item[Quantitative Literacy] -- Students will use chemical equations, graphs and tables to interpret and communicate chemical information.
	\item[Inquiry and Analysis] -- Students will investigate chemical problems.
	\item[Communication] -- Students will report laboratory results clearly and concisely.
	\item[Problem Solving] -- Students will implement experimental procedures.
	\item[Teamwork] -- Students will productively interact with each other to successfully conduct chemistry experiments.
\end{description}

\section*{Laboratory Work}
Before lab, you are expected to have read the handout of your experiment as well as review your lecture notes from class. Come prepared to enter your data into the lab computers and have a USB drive with you. You may perform each laboratory with a lab partner and you may acquire your data together during your scheduled lab time. However, you must NOT work with your lab partner beyond this. All analysis of data and calculations as well as all laboratory reports must be done on an individual basis. Failure to do so will result in a zero for the lab in question.

\noindent Please follow all safety procedures, especially by wearing safety glasses or goggles. When leaving the lab, please make sure it is in the same condition as it was when you arrived. Be respectful of others.

\paragraph{Laboratory Risk:}
Chemical exposure is a constant risk in a chemistry lab. To minimize the risk to yourself and those around you, the following rules must be followed:
\begin{itemize}
	\item Never taste or smell a chemical or pipette by mouth.
	\item Wash your hands before leaving the lab and frequently during the lab to avoid accidental contamination of yourself and others.
	\item Dispose of chemicals only as directed. Nothing goes down the sink unless expressly directed.
	\item Keep your work area clean; wipe up any spills (liquid or solid) immediately.
	\item Replace caps on reagent bottles, and never return chemicals to the original container.
	\item No shorts, tank tops, or sandals allowed in lab, and long hair should be restrained.
	\item Wear safety glasses at all times when in the lab.
\end{itemize}
Students enrolling in this course should realize that they are voluntarily exposing themselves to a variety of chemicals, some of which could be irritating or hazardous with excessive exposure.  For those persons with a sensitive medical condition such as allergies, precautions such as wearing additional protective garments, delaying enrolling, or even not enrolling in a class may be necessary.

\section*{Tentative Schedule}
This class will meet Thursdays from 3:00 to 5:50 pm in room 208 of the Skaggs Center for Health \& Molecular Sciences (SCA)

\paragraph{Week 1: Jan. 13}~

\textbf{No lab this week!}

\paragraph{Week 2: Jan. 20}~

-- Check-in and safety discussion

-- Intermolecular Forces

\paragraph{Week 3: Jan. 27}~

-- Vitamin C Titration

\paragraph{Week 4: Feb. 3}~

-- Freezing Points of Solutions

\paragraph{Week 5: Feb. 10}~

-- Reaction Rates

\paragraph{Week 6: Feb. 17}~

-- Rate Laws

\paragraph{Week 7: Feb. 24}~

-- Le Chatelier

\paragraph{Week 8: Mar. 3}~

\textbf{No lab -- Spring Break!}

\paragraph{Week 9: Mar. 10}~

-- Acids and Bases

\paragraph{Week 10: Mar. 17}~

-- Buffers

\paragraph{Week 11: Mar. 24}~

-- Acid Base Titrations

\paragraph{Week 12: Mar. 31}~

\textbf{No lab -- Festival of Excellence!}

\paragraph{Week 13: Apr. 7}~

-- Electrochemistry

\paragraph{Week 14: Apr. 14}~

-- Electrolysis and Plating

\paragraph{Week 15: Apr. 21}~

-- Lab Checkout and Final Exam -- Come as normally scheduled for administration of the final exam. Bring a calculator and any writing utensil (or two).


\section*{Course Requirements}
Grades will be based on the following items:
\begin{description}
  \item[Pre-Lab Quizzes] 10 Points Each
  \item[Safety and Clean-up] 5 Points Each
  \item[Lab Reports] 30 Points Each
  \item[Final Exam] 200 Points
\end{description}
Final Grades will be assigned according to the following scale:

\begin{tabular}{rl|c|rl}
	Percentage & Grade &  & Percentage & Grade \\ \midrule
	  93.0-100 & A     &  &  73.0-77.0 & C     \\
	 90.0-93.0 & A-    &  &  70.0-73.0 & C-    \\
	 87.0-90.0 & B+    &  &  67.0-70.0 & D+    \\
	 83.0-87.0 & B     &  &  63.0-67.0 & D     \\
	 80.0-83.0 & B-    &  &  60.0-63.0 & D-    \\
	 77.0-80.0 & C+    &  &     < 60.0 & F
\end{tabular}
\paragraph{Pre-Lab Quizzes:}
These quizzes will be posted on the Canvas course website and must be completed before the start of lab each week.

\paragraph{Safety and Clean-up:}
Any student who violates laboratory rules or engages in unsafe behavior in the laboratory may lose points. Any student who leaves their station without fully cleaning up after the lab period will likewise lose points.

\paragraph{Lab Reports:}
Lab report pages are included at the end of the instructional material for each lab. These reports are due at the beginning of the next scheduled laboratory day.

\paragraph{Final Exam:}
The final exam is comprehensive and questions will draw on chemical concepts, laboratory techniques, results, and analysis and interpretation.

\paragraph{Attendance Policy:}
Students are expected to attend class. If you must miss class, contact the instructor ahead of time.

\paragraph{Late Work Policy:}
All pre-lab quizzes must be completed before the start of each lab period, and all reports are to be turned in at the \emph{beginning} of the lab period. Late work will not be accepted.

\paragraph{Make-up Work Policy:}
In general, there will be no opportunity to make up missed work. If you must miss class, please contact the instructor ahead of time.

\section*{Miscellany}

\paragraph{Scientific Calculator:}
There are many different ways to calculate figures during homework. It is tempting to rely on Online resources such as \href{http://www.wolframalpha.com}{http://www.wolframalpha.com}, or to simply use a calculator application on a smart phone. During exams, however, any devices capable of connecting to the Internet will \emph{not} be allowed. You will instead need a scientific calculator capable of performing exponentiation and logarithms for the exams. Using this calculator exclusively while doing homework will ensure that you are familiar with it for use during exams.

\paragraph{Academic Integrity:}
Scholastic dishonesty will not be tolerated and will be prosecuted to the fullest extent. You are expected to have read and understood the current issue of the \href{https://help.suu.edu/handbook}{Student Handbook} (published by Student Services) regarding student responsibilities and rights, and for the intellectual property policy, information about procedures, and what constitutes acceptable behavior. From University policy 6.33: ``The University defines plagiarism as intentionally or carelessly presenting the work of another as one’s own. It includes submitting an assignment purporting to be the student’s original work which has wholly or in part been created by another person, or cutting and pasting of source material\ldots''

\paragraph{ADA Policy:}
Students with medical, psychological, learning, or other disabilities desiring academic adjustments, accommodations, or auxiliary aids will need to contact the Southern Utah University Coordinator of Services for Students with Disabilities (SSD), in Room 206F of the Sharwan Smith Center or phone (435) 865-8022. SSD determines eligibility for and authorizes the provision of services.

\paragraph{Emergency Management Statement:}
In case of emergency, the university's Emergency Notification System (ENS) will be activated. Students are encouraged to maintain updated contact information using the link on the homepage of the \emph{mySUU} portal. In addition, students are encouraged to familiarize themselves with the Emergency Response Protocols posted in each classroom. Detailed information about the university's emergency management plan can be found at: \href{http://www.suu.edu/emergency}{http://www.suu.edu/emergency}

\paragraph{HEOA Compliance Statement:}
The sharing of copyrighted material through peer-to- peer (P2P) file sharing, except as provided under U.S. copyright law, is prohibited by law. Detailed information can be found at: \href{https://help.suu.edu/article/1097/p2p-and-copyright-infringement}{https://help.suu.edu/article/1097/p2p-and-copyright-infringement}

\paragraph{LINK Statement:}
SUU faculty and staff care about the success of our students. In addition to your professor, numerous services are available to assist you with the achievement of your educational goals. SUU's LINK system may be used by faculty to notify you and/or your advisors of their concern for your progress and provide references to campus services as appropriate.

\paragraph{SUUSA Statement:}
As a student at SUU, you have representation from the SUU Student Association (SUUSA) which advocates for student interests and helps work as a liaison between the students and the university administration. You can submit My SUU Voice feedback by going here: \href{https://www.suu.edu/suusa/voice}{https://www.suu.edu/suusa/voice} Likewise, you can learn more about SUUSA's Executive Council here (\href{https://www.suu.edu/suusa/executive-council/}{https://www.suu.edu/suusa/executive-council/}) and about indivdual SUUSA's Student Sentors here (\href{https://www.suu.edu/suusa/senate/}{https://www.suu.edu/suusa/senate/})

\paragraph{University Policies and Recommendations Regarding COVID-19:}
Southern Utah University has compiled a collection of information, policies, and recommendations related to COVID-19 at \href{https://www.suu.edu/coronavirus/}{suu.edu/coronavirus/}

\noindent I dearly want this semester to go smoothly vis-\`a-vis COVID-19, and I assume you all do as well. Toward that end, I encourage you all to exercise all reasonable precaution to prevent the spread of the coronavirus. This includes using the testing and self-reporting resources at the link above.

\noindent It may interest some of you to know that, for my part, I have been vaccinated with two doses of the Moderna vaccine, and more recently received a Pfizer booster. I will wear a mask on campus when appropriate. I will \emph{not} be wearing a mask as I lecture, since clear communication is my primary goal in the classroom.

\paragraph{Disclaimer:}
Information contained in this syllabus, other than the grading, late assignments, make up work and attendance policies, may be subject to change as deemed appropriate by the instructor.

\end{document}
